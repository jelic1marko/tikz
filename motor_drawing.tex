\documentclass[a4paper, 11pt]{article} 
\usepackage[top=3cm, bottom=3cm, left=2.75cm, right=2.75cm]{geometry}

% ---------------------------------------
% Induction and DC motor drawings
% By Marko Jelić, marko95jelic@gmail.com
% ---------------------------------------

% language settings
\usepackage[english]{babel}
\usepackage[utf8]{inputenc} 
\usepackage[OT1]{fontenc} 

% figures, graphics & plots
\usepackage[pdftex]{graphicx}
\usepackage{caption}
\usepackage{subcaption} 
\usepackage{pgfplots}
\usepackage{tikz}
	\usetikzlibrary{calc,patterns,decorations.pathmorphing,decorations.markings}
	\usetikzlibrary{arrows, shapes, calc, positioning}
	\usetikzlibrary{circuits.logic.US}
	\usetikzlibrary{shadings, calc, decorations.markings}
    \tikzset{->-/.style={decoration={
      markings,
      mark=at position #1 with {\arrow{>}}},postaction={decorate}},
      ->-/.default=0.5,
      }
    \tikzset{
        partial ellipse/.style args={#1:#2:#3}{
            insert path={+ (#1:#3) arc (#1:#2:#3)}
        }
    }
\usepackage[siunitx]{circuitikz}
    
% styles
\usepackage{tcolorbox}
\usepackage{color, colortbl}
\definecolor{Gray}{gray}{0.9}

% math packages
\usepackage{amsmath} 
\usepackage{amsfonts} 
\usepackage{amssymb} 
\usepackage{mathtools}

% define arc
\def\centerarc[#1](#2)(#3:#4:#5)% Syntax: [draw options] (center) (initial angle:final angle:radius)
    { \draw[#1] ($(#2)+({#5*cos(#3)},{#5*sin(#3)})$) arc (#3:#4:#5); }
 \usepackage{soul}
 
\begin{document}

\begin{figure}[h!]
    \centering
	\begin{subfigure}[t]{0.49\textwidth}
	    \centering
        \begin{circuitikz}[scale = 0.9]
        
            % Draw coordinate system/grid for easier positioning
            % \draw[help lines,step=10mm,gray!20] (-3,-3) grid (4,10);
            % \draw[-latex, line width=0.25mm, thick, dashed] (-3,0) -- (4,0);
            % \draw[-latex, line width=0.25mm, thick, dashed] (0,-3) -- (0,10);
        
            % Define a formula for the coil around the core
            \def\coil#1{
                {1 * cos(\t * pi r)},
                {0.165 * (1.8*#1 + \t) + 0.4*sin(\t * pi r))}
                }
                
            % Draw the part of the coil behind the rectangle
            \foreach \n in {0,1,...,16} {
                \draw[domain={0:1},smooth,variable=\t,samples=15]
                    plot (\coil{\n}); 
                }
                
            % Draw the rectangle/iron core
            \fill [gray!40, rounded corners=5pt] (-0.8,-1) rectangle (0.8,8);
            \node at (-0.5,8.5) [] {$\mu$};
            
            % Draw the part of the coil in front of the rectangle
            \foreach \n in {0,1,...,16} {
                \draw[domain={1:2},smooth,variable=\t,samples=15,
                      preaction={draw,white,line width=3pt}     % remove if undesired
                     ]
                    plot (\coil{\n});
                }
            
            % Draw torus at the top of the core    
            \filldraw [draw=black, fill=white] (0,7) ellipse (2 and 0.7);
            \draw [] (0,7) [partial ellipse=10:170:1.2 and 0.3];
            \fill [gray!40, rounded corners=5pt] (-0.8,6.7) rectangle (0.8,9);
            \draw [preaction={draw,white,line width=3pt}] (0,7.1) [partial ellipse=160:380:1.25 and 0.4];

            % Draw arrows denoting the magnetic induction from the core
            \draw [-stealth, ultra thick] (-2,7) [partial ellipse=0:35:1.1 and 4];
            \draw [-stealth, ultra thick] (-2,7) [partial ellipse=0:10:1.75 and 15];
            \draw [stealth-, ultra thick] (2,7) [partial ellipse=145:180:1.1 and 4];
            \draw [stealth-, ultra thick] (2,7) [partial ellipse=170:180:1.75 and 15];
            \node at (1.5,9) [] {$\pmb{B}_\mathrm{c}$};

            % Draw the circuit that feeds current to the coil around the core
            \draw (1,5) to[short, *-] (2.5,5) to[american controlled current source, l_=$I_\mathrm{g}$] (2.5,0) to[short, -*] (1,0);
            \draw [dashed, ] (0,5.5) -- (2.5,5.5);
            \draw [dashed, shorten <= 0.1cm] (2,7) -- (2.5,7);
            \node at (2.5,7.5) [] {$R_\mathrm{r}$, $L_\mathrm{r}$};
            \node at (-2,5) [] {$R_\mathrm{c}$, $L_\mathrm{c}$};
            
            % Draw the implicit positional control for the torus
            \draw [stealth-stealth, shorten <= 0.1cm, shorten >= 0.1cm] (2.5,5.5) -- (2.5,7) node [midway, right] {$z$};
            \draw [dotted, thick] (3,6.25) -- (4,6.25) -- (4,2.5);
            \draw [-stealth, dotted, thick, shorten >= 0.1cm] (4,2.5) -- (3,2.5);
            
        \end{circuitikz}
		\caption{linear motor}
	\end{subfigure}
    % \quad
	\begin{subfigure}[t]{0.49\textwidth} 
		\centering
        \begin{circuitikz}[scale = 0.9]
        
            % Draw coordinate system/grid for easier positioning
            % \draw[help lines,step=10mm,gray!20] (-3,-3) grid (4,10);
            % \draw[-latex, line width=0.25mm, thick, dashed] (-3,0) -- (4,0);
            % \draw[-latex, line width=0.25mm, thick, dashed] (0,-3) -- (0,10);
        
            % Define a formula for the coil around the lower core
            \def\coil#1{
                {1 * cos(\t * pi r)},
                {0.165 * (1.8*#1 + \t) + 0.4*sin(\t * pi r))}
                }   
            % Draw the part of the coil behind the lower rectangle
            \foreach \n in {0,1,...,4} {
                \draw[domain={0:1},smooth,variable=\t,samples=15]
                    plot (\coil{\n}); 
                }
            % Draw the lower rectangle/iron core
            \fill [gray!40, rounded corners=5pt] (-0.8,-0.5) rectangle (0.8,2.0);
            % Draw the part of the coil in front of lower the core
            \foreach \n in {0,1,...,4} {
                \draw[domain={1:2},smooth,variable=\t,samples=15,
                      preaction={draw,white,line width=3pt}     % remove if undesired
                     ]
                    plot (\coil{\n});
                }
                
            % Define a formula for the coil around the upper core
            \def\coil#1{
                {1 * cos(\t * pi r)},
                {6 + 0.165 * (1.8*#1 + \t) + 0.4*sin(\t * pi r))}
                }                    
            % Draw the part of the coil behind the upper core
            \foreach \n in {0,1,...,4} {
                \draw[domain={0:1},smooth,variable=\t,samples=15]
                    plot (\coil{\n}); 
                }            
            % Draw the upper rectangle/iron core
            \fill [gray!40, rounded corners=5pt] (-0.8,5.5) rectangle (0.8,8.0);
            \node at (-0.5,7.5) [] {$\mu$};
            % Draw the part of the coil in front of the upper rectangle
            \foreach \n in {0,1,...,4} {
                \draw[domain={1:2},smooth,variable=\t,samples=15,
                      preaction={draw,white,line width=3pt}     % remove if undesired
                     ]
                    plot (\coil{\n});
                }
            
            % Draw arrows denoting the magnetic induction from the core
            \draw [-stealth, ultra thick] (-2,3.75) [partial ellipse=-22:22:1.5 and 4];
            \draw [-stealth, ultra thick] (-2,3.75) [partial ellipse=-6:6:1.85 and 15];
            \draw [stealth-, ultra thick] (2,3.75) [partial ellipse=158:202:1.5 and 4];
            \draw [stealth-, ultra thick] (2,3.75) [partial ellipse=174:186:1.85 and 15];
            \node at (-1,5) [] {$\pmb{B}_\mathrm{s}$};
            
            % Draw the passive rotor (short circuited inductor)
            \draw (-1,2.75) to[/tikz/circuitikz/bipoles/length=80pt, L] (1,4.75);
            \draw (-0.7,2.45) -- (1.3,4.45); 
            \draw (-1,2.75) -- (-0.7,2.45);
            \draw (1,4.75) -- (1.3,4.45);
            
            % Draw the implicit positional control for the torus
            \draw [dashed, shorten <= -1cm, shorten >= -0.5cm] (0,3.75) -- (2,5.75);
            \draw [dashed, shorten <= -0cm, shorten >= -1cm] (0,3.75) -- (2,3.75);
            \draw [-stealth, shorten <= 0.1cm, shorten >= 0.1cm] (0,3.75) [partial ellipse=0:45:2.9 and 2.9];
            \draw [thick, dotted] (3.25,5) -- (4,5) -- (4,2.5);
            \draw [thick, dotted, shorten >= 0.1cm, -stealth] (4,2.5) -- (4.5,2.5);
            \node at (3,5) [] {$\theta$};
            \node at (-1.25,4) [] {$R_\mathrm{r}$, $L_\mathrm{r}$};
            
            % Draw the circuit that feeds current to the coils around the cores
            \draw (1,1.5) to[short, *-] (2,1.5) -- (2,6) to[short, -*] (1,6);
            \draw (1,7.5) to[short, *-] (5,7.5) -- (5,5) to[american controlled current source, l=$I\mathrm{g}$] (5,0) to[short, -*] (1,0);
            \node at (2.15,0.75) [] {$R_\mathrm{s}/2$, $L_\mathrm{s}/2$};
            \node at (2.15,6.75) [] {$R_\mathrm{s}/2$, $L_\mathrm{s}/2$};            
            
        \end{circuitikz}
		\caption{rotary motor}
	\end{subfigure}
	\caption{Induction motor}
	\label{fig:pvr}
\end{figure}


\begin{figure}[h!]
	\centering
	\begin{subfigure}[t]{0.49\textwidth}
	    \centering
        \begin{circuitikz}[scale = 0.9]
        
            % Draw coordinate system/grid for easier positioning
            % \draw[help lines,step=10mm,gray!20] (-3,-5) grid (5,8);
            % \draw[-latex, line width=0.25mm, thick, dashed] (-3,0) -- (5,0);
            % \draw[-latex, line width=0.25mm, thick, dashed] (0,-5) -- (0,8);
        
            % Define a formula for the coil.
            \def\coil#1{
                {1 * cos(\t * pi r)},
                {0.165 * (1.8*#1 + \t) + 0.4*sin(\t * pi r))}
                }
            % Draw the part of the coil behind the rectangle
            \foreach \n in {0,1,...,16} {
                \draw[domain={0:1},smooth,variable=\t,samples=15]
                    plot (\coil{\n}); 
                }
            % Draw the rectangle
            \fill [gray!40, rounded corners=5pt] (-0.8,-1) rectangle (0.8,6);
            % Draw the part of the coil in front of the rectangle
            \foreach \n in {0,1,...,16} {
                \draw[domain={1:2},smooth,variable=\t,samples=15,
                      preaction={draw,white,line width=3pt}     % remove if undesired
                     ]
                    plot (\coil{\n});
                }
            
             % Draw arrows denoting the magnetic induction from the core
            \draw [-stealth, ultra thick] (-2,-3) [partial ellipse=-10:35:1.5 and 4];
            \draw [-stealth, ultra thick] (-2,-3) [partial ellipse=-5:10:1.75 and 15];
            \draw [stealth-, ultra thick] (2,-3) [partial ellipse=145:190:1.5 and 4];
            \draw [stealth-, ultra thick] (2,-3) [partial ellipse=170:185:1.75 and 15];
            \node at (-1.25,-0.75) [] {$\pmb{B}_\mathrm{c}$};
            
            % Draw the circuit that feeds current to the coil around the core
            \draw (1,5) to[short, *-] (2.5,5) to[american controlled current source, l_=$I_\mathrm{g}$] (2.5,0) to[short, -*] (1,0);
            \draw [dashed, ] (0,-0.25) -- (3,-0.25);
            \draw [dashed, shorten <= 0.1cm] (0.2,-2.9) -- (3,-2.9);
            \draw [stealth-stealth, shorten <= 0.1cm, shorten >= 0.1cm] (3,-0.25) -- (3,-2.9) node [midway, right] {$z$};
            \node at (-2,5) [] {$R_\mathrm{c}$, $L_\mathrm{c}$};
                            
            % Draw the permanent magnet and its lines
            \draw [dotted, thick] (3.5,-1.575) -- (4,-1.575) -- (4,2.5);
            \draw [-stealth, dotted, thick, shorten >= 0.1cm] (4,2.5) -- (3,2.5);
            \path [use as bounding box] (-3.2,-3.2) rectangle (3.2,3.2);
            \node[draw, left color=blue!60,right color=red!60,  middle color=white, 
                rotate=90, shading angle=90+90, minimum width=0.75cm,
                ]  at (0,-2.5) (I) {};
            \draw[->-, gray, thick] (I.south east) .. controls (0.75,-0.5) and (0.75,-4.5) ..  (I.south west);    
            \coordinate (n1) at ($ (I.south east)!0.5!(I.east) $);
            \coordinate (s1) at ($ (I.south west)!0.5!(I.west) $);  
            \draw[->-, gray, thick] (n1) .. controls (1.5, 0.25) and (1.5, -5.25) .. (s1);
            \draw[->-, gray, thick] (I.east) .. controls (2.5,1.5) and (2.5,-6.5) .. (I.west);
            \draw[->-, gray, thick] (I.east) .. controls (-2.5,1.5) and (-2.5,-6.5) .. (I.west);
            \coordinate (n2) at ($ (I.north east)!0.5!(I.east) $);
            \coordinate (s2) at ($ (I.north west)!0.5!(I.west) $);
            \draw[->-, gray, thick] (n2) .. controls (-1.5, 0.25) and (-1.5, -5.25) .. (s2);
            \draw[->-, gray, thick] (I.north east) .. controls (-0.75,-0.5) and (-0.75,-4.5) ..  (I.north west);
            \node at (2,-1.25) [] {$\pmb{B}_\mathrm{m}$};
            
            \vspace{-40pt}
        \end{circuitikz}
		\caption{linear motor}
	\end{subfigure}
	%\quad 
	\begin{subfigure}[t]{0.49\textwidth} 
		\centering
        \begin{circuitikz}[scale = 0.9]
            
            % Draw coordinate system/grid for easier positioning
            % \draw[help lines,step=10mm,gray!20] (-2,-2) grid (7,9);
            % \draw[-latex, line width=0.25mm, thick, dashed] (-2,0) -- (7,0);
            % \draw[-latex, line width=0.25mm, thick, dashed] (0,-2) -- (0,9);    
            
            % Define a formula for the coil around the lower core 
            \def\coil#1{
                {1 * cos(\t * pi r)},
                {0.165 * (1.8*#1 + \t) + 0.4*sin(\t * pi r))}
                }
            % Draw the part of the coil behind the rectangle
            \foreach \n in {0,1,...,4} {
                \draw[domain={0:1},smooth,variable=\t,samples=15]
                    plot (\coil{\n}); 
                }
            % Draw the lower rectangle/iron core
            \fill [gray!40, rounded corners=5pt] (-0.8,-0.5) rectangle (0.8,2.0);
            % Draw the part of the coil in front of lower the core
            \foreach \n in {0,1,...,4} {
                \draw[domain={1:2},smooth,variable=\t,samples=15,
                      preaction={draw,white,line width=3pt}     % remove if undesired
                     ]
                    plot (\coil{\n});
                }
            \node at (2.15,0.75) [] {$R_\mathrm{s}/2$, $L_\mathrm{s}/2$};    
            \node at (1,2) [red] {$N$};
                
            % Define a formula for the coil around the upper core
            \def\coil#1{
                {1 * cos(\t * pi r)},
                {6+0.165 * (1.8*#1 + \t) + 0.4*sin(\t * pi r))}
                }                    
            % Draw the part of the coil behind the upper core
            \foreach \n in {0,1,...,4} {
                \draw[domain={0:1},smooth,variable=\t,samples=15]
                    plot (\coil{\n}); 
                }
            % Draw the upper rectangle/iron core
            \fill [gray!40, rounded corners=5pt] (-0.8,5.5) rectangle (0.8,8.0);
            % Draw the part of the coil in front of the upper rectangle
            \foreach \n in {0,1,...,4} {
                \draw[domain={1:2},smooth,variable=\t,samples=15,
                      preaction={draw,white,line width=3pt}     % remove if undesired
                     ]
                    plot (\coil{\n});
                }
            \node at (2.15,6.75) [] {$R_\mathrm{s}/2$, $L_\mathrm{s}/2$};
            \node at (-0.5,7.5) [] {$\mu$};    
            \node at (1,5.5) [blue] {$S$};    
            
            Draw the short connection between the lower and upper coils
            \draw (1,1.5) to[short, *-] (2,1.5) -- (2,6) to[short, -*] (1,6);
            
             % Draw the active rotor
            \draw (-1,2.75) to[/tikz/circuitikz/bipoles/length=80pt, L] (1,4.75);
            \draw (-0.5,2.25) to[european voltage source] (1.5,4.25); 
            \draw (-1,2.75) -- (-0.5,2.25);
            \draw (1,4.75) -- (1.5,4.25);
            \node at (-1.25,4) [] {$R_\mathrm{r}$, $L_\mathrm{r}$};
            \node at (1.25,3) [] {$e_\mathrm{r}$};
            \node at (1.1,3.5) [] {$+$};
            
            % Draw arrows denoting the magnetic induction from the rotor
            \draw [stealth-, gray, thick, shorten <= -0.25cm, shorten >= -1cm] (-1.25,2.4) -- (0.75,4.4);
            \draw [stealth-, gray, thick, shorten <= -0.25cm, shorten >= -1cm] (-1.25,2.6) -- (0.75,4.6);
            \node at (-1.5,3) [] {$\pmb{B}_\mathrm{r}$};
            
            % Draw arrows denoting the magnetic induction from the core
            \draw [-stealth, ultra thick] (-2,3.75) [partial ellipse=-6:6:1.85 and 15];
            \draw [stealth-, ultra thick] (2,3.75) [partial ellipse=174:186:1.85 and 15];
            \node at (-0.5,5) [] {$\pmb{B}_\mathrm{s}$};
            
             % Draw the implicit positional control for the torus
            \draw [dashed, shorten <= -1cm, shorten >= -0.5cm] (0,3.75) -- (2,5.75);
            \draw [dashed, shorten <= -0cm, shorten >= -1cm] (0,3.75) -- (2,3.75);
            \draw [-stealth, shorten <= 0.1cm, shorten >= 0.1cm] (0,3.75) [partial ellipse=0:45:2.9 and 2.9];
            \node at (3,5) [] {$\theta$};
            
            % Draw the circuit that feeds current to the coils around the cores
            \draw (1,7.5) to[short, *-] (5,7.5) -- (5,5) to[american controlled current source, l=$I\mathrm{g}$] (5,0) to[short, -*] (1,0);
            \draw [thick, dotted] (3.25,5) -- (4,5) -- (4,2.5);
            \draw [thick, dotted, shorten >= 0.1cm, -stealth] (4,2.5) -- (4.5,2.5);
    
        \end{circuitikz}
		\caption{rotary motor}
	\end{subfigure}
	\caption{DC motor}
	\label{fig:dvr}
\end{figure}

% bibliography:
% https://tex.stackexchange.com/questions/315292/tikz-magnetic-field
% https://tex.stackexchange.com/questions/32297/modify-tikz-coil-decoration/43605

\end{document}
